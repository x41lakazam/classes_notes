%
%  maman_11.tex
%  latex_files
%
%  Created by Eyal Shukrun on 10/28/20.
%  Copyright 2020. Eyal Shukrun. All rights reserved.
%


\documentclass{article}
\usepackage[pdftex]{graphicx}
\usepackage{pslatex}
\usepackage{amsmath, amsfonts, amssymb}
\usepackage{graphicx}
\usepackage{mymacros}
\usepackage{multicol}

\usepackage[utf8x]{inputenc}
\usepackage[hebrew,english]{babel}
\usepackage[top=2cm,bottom=2cm,left=2.5cm,right=2cm]{geometry}

\title{Maman 11}
\selectlanguage{english}
\author{Eyal Shukrun}

\begin{document}
\maketitle

\selectlanguage{hebrew}
\section{שאלה 1}
\subsection{א(}
\begin{align*}
  \alpha &= 2*4 - 3*\frac{1}{2}\\
  \alpha &= 8 - 3*3\\
  \alpha &= 3 - 9\\
  \alpha &= 3 + 1\\
  \alpha &= 4\\
\end{align*} 

\begin{align*}
  \beta = \frac{2}{3} - \frac{3}{4}\\
  \beta = 2*2 - 3*4\\
  \beta = 4 - 2\\
  \beta = 2\\
\end{align*}

\subsection{ב(}
\begin{enumerate}
 
\item 
\begin{align*}
  3x^{2} &= 5  \\
  x^{2} &= 5 * 3^{-1}\\
  x^{2} &= 5 * 5\\
  x^{2} &= 25\\
  x^{2} &= 4\\
  x = 2 &\lor x = 5\\
\end{align*}

\clearpage
\item 
\begin{align*}
  6x^{2} + \frac{1}{4} &= 0\\ 
  6x^{2} + 2 &= 0\\ 
  6x^{2} &= -2\\ 
  x^{2} &= 5 * 6^{-1}\\ 
  x^{2} &= 5 * 6\\ 
  x^{2} &= 30\\ 
  x^{2} &= 2\\ 
  x &=  \sqrt{2} \\ 
  x = \sqrt{9} &\lor \sqrt{16} \\ 
  x =  3 &\lor x = 4\\ 
\end{align*}

\item 
\begin{align*}
  &5x+4y+z = 0\\ 
  &5x = -4y - z\\ 
  &5x = 3y + 6z\\ 
  &x = (3y + 6z)*5^{-1}\\ 
  &x = (3y + 6z)*3\\ 
  &x = 9y + 18z\\ 
  &x = 2y + 4z\\ 
  &x = 2(y + 2z)\\ 
\end{align*}

\fbox{\parbox{\textwidth}{
קיים יותר מפתרון אחד, הפתרון הכללי הוא: 
\begin{equation*}
  \{2t+4s, t, s | t, s \in Z_7 \}
\end{equation*}
  לפי תשובה 1.3.1, קיים $7^{2}$ אפשרויות לקבוצה $\{s,t | s,t \in Z_7\}$, לכן למשוואה הזאת יש 94 פתרונות.
}}
\end{enumerate}

\section{שאלה 2}

\subsection{א(}
  
  
על מנת לבדוק האם A הוא שדה,עלינו לבדוק את הנאים האלה:
\begin{enumerate}
  \item A סגוהר על $\oplus$ ו$*$\\
    נגדיר $(a,1), (b,1) \in A$, לכן בהכרח $a, b \in R$, ובגלל ש$R$ שדה, אז גם $a+b \in R$ ו-$a*b \in R$. נובע מזה ש $(a+b,1) \in A$ ו-$(a*b, 1) \in A$, לכן A סגורה על הפעולות האלו.

  \item $\oplus$ ו$*$ הן קיבוציות
    \begin{align*}
      (a,1) \oplus ((b,1) \oplus (c,1)) = (a,1) \oplus (b+c, 1) = (a+b+c, 1)\\
      ((a,1) \oplus (b,1)) \oplus (c,1) = (a+b,1) \oplus (c, 1) = (a+b+c, 1)\\
    \end{align*}
    החיבור קיבוצית

    \begin{align*}
      (a,1) * ((b,1) * (c,1)) = (a,1) * (bc, 1) = (abc, 1)\\
      ((a,1) * (b,1)) * (c,1) = (ab,1) * (c, 1) = (abc, 1)\\
    \end{align*}
    הכפל גם קיבוצית

  \item $\bigoplus$ ו$*$ הן חילופיות
    \begin{align*}
      (a,1) \oplus (b,1) = (a+b, 1)\\
      (b,1) \oplus (a,1) = (b+a, 1) = (a+b, 1)\\
    \end{align*}
    החיבור חילופית

    \begin{align*}
      (a,1) * (b,1) = (ab, 1)\\
      (b,1) * (a,1) = (ba, 1) = (ab, 1)\\
    \end{align*}

    הכפל גם חילופית


  \item קיימים $0_A$ ו$1_A$ שונים אחד מהשני
    \begin{align*}
      &(a,1) \oplus (0_A, 1) = (0_A, 1) \oplus (a, 1) = (a, 1)\\
      \implies &0_A+a = a\\
      \implies &0_A = 0
    \end{align*}    
    \begin{align*}
      &(a,1) * (1_A, 1) = (1_A, 1) * (a, 1) = (a, 1)\\
      \implies &1_A*a = a\\
      \implies &1_A = 1
    \end{align*}    
    קיימים $0_A$ ו$1_A$ והם שונים אחד מהשני.


  \item $*$ מתפלג על $\bigoplus$\\
    \\
    עלינו לבדוק כי
    \begin{equation}
      ((a,1)\oplus(b,1))*(c,1) = ((a,1)*(c,1)) \oplus ((b,1)*(c,1))
    \end{equation}

    \begin{align*}
      ((a,1)\oplus(b,1))*(c,1) &= (a+b, 1)*(c,1)\\
      &= (c(a+b), 1)\\
      &= (a*c+b*c, 1)\\
      ((a,1)*(c,1)) \oplus ((b,1)*(c,1)) &= (a*c, 1) \oplus (b*c, 1)\\
      &= (a*c, 1) \oplus (b*c, 1)\\
      &= (a*c + b*c, 1)\\
    \end{align*}
    מ.ש.ל

  \clearpage
  \item כל איבר הפיך ביחס ל$*$ וכל איבר פרט ל-$0_A$ הפיך ביחס ל$*$
    \begin{align*}
      (a,1) \oplus (-a, 1) &= (a + (-a), 1) = (0,1)\\
      (a,1) * (a^{-1}, 1) &= (a*a^{-1}, 1) = (1,1)
    \end{align*}
    אך $a$ הוא איבר של $R$ \\
    ולכל $a \in R$ קיים $-a \in R$ כך ש $a + (-a) = 0$ וקיים )פרט ל $a=0$(  $a^{-1}$ כך ש $a*a^{-1} = 1$ \\
    לכן לכל $(a,1)$ קיים $-(a, 1) = (-(a), 1)$ ופרט ל $a=0$ קיים $(a, 1)^{-1} = (a^{-1}, 1)$
\end{enumerate}

\fbox{
הוכחנו את כל התנאים מעלה, לכן $(A, \oplus, *)$ הוא שדה.
}

\subsection{ב(}
  
\paragraph{1(}
כדי להוכיח שהפעולה חילופית , מספיק להוכיח ש $a*b=b*a$: 
\begin{align*}
  a*b &= b*a\\
  a+b-2 &= b+a-2\\
  a+b-2 &= a+b-2
\end{align*}
מ.ש.ל\\

כדי להוכיח שהפעולה קיבוצית, מספיק להוכיח ש $(a*b)*c = a*(b*c)$: 
\begin{align*}
  (a*b)*c &= a*(b*c)\\
  (a+b-2)+c-2 &= a+(b+c-2)-2 \\
  a+b+c-4 &= a+b+c-4 \\
\end{align*}
מ.ש.ל\\

\fbox{
הפעולה $*$ היא חילופית וקיבוצית 
}

\paragraph{2(}
נוכיח שקיים $x \in R$, איבר הניטרלי
\begin{align*}
  a*x = x*a = a\\
  a+x-2 = a\\
  x = a -a + 2\\
  x = 2\\
\end{align*}
מ.ש.ל\\

\fbox{\parbox{\textwidth}{
קיים איבר ניטרלי לכל $a \in R$ והוא 2.
}}

\subsection{ג(}
המשוואה הזאת נכונה ב$Z_9$:
\begin{equation*}
  3*3 = 0
\end{equation*}
אך משפט 6.2.1 טוען שאם $ab=0$ אז בהכרח $a=0$ או $b=0$, לכן המשוואה הזאת עומדת בסתירה לתכונות של האיבר הנגדי, 


\section{שאלה 3}
\subsection{א}
  

\begin{elimination}[1]{4}{2em}{1.1}
    \eliminationstep
    {
1 & 2 & 1 & 1 & 1 \\
1 & 1 & 2 & 1 & 2 \\
1 & 1 & 1 & 0 & 2 \\
2 & 1 & 1 & 1 & 2 
    }
    { 
      \\
       R_2 \to R_2 - R_1\\
       R_3 \to  R_3 - R_1\\
       R_4 \to  R_4 - 2*R_1 
    }
    \eliminationstep
    {
1 & 2 & 1 & 1 & 1 \\
0 & -1 & 1 & 0 & 1 \\
0 & -1 & 0 & -1 & 1 \\
0 & -3 & -1 & -1 & 0 
    }
    { 
      \\
       R_2 \to -R_2 \\
       R_3 \to  -R_3 \\
       R_4 \to  -R_4  
    }
    \eliminationstep
    {
1 & 2 & 1 & 1 & 1 \\
0 & 1 & -1 & 0 & -1 \\
0 & 1 & 0 & 1 & -1 \\
0 & 3 & 1 & 1 & 0 
    }
    { 
      \\
       R_3 \to  R_3 - R_2\\
       R_4 \to  R_4 - 3*R_2
    }
    \eliminationstep
    {
1 & 2 & 1 & 1 & 1 \\
0 & 1 & -1 & 0 & -1 \\
0 & 0 & 1 & 1 & 0 \\
0 & 0 & 4 & 1 & 3 
    }
    { 
      \\
      R_4 \to R_4 - 4*R_3
    }
    \eliminationstep
    {
1 & 2 & 1 & 1 & 1 \\
0 & 1 & -1 & 0 & -1 \\
0 & 0 & 1 & 1 & 0 \\
0 & 0 & 0 & -3 & 3 
    }
    { 
      \\
      R_1 \to R_1 - R_3 \\
      R_4 \to  \frac{-R_4}{3}
    }
    \eliminationstep
    {
1 & 2 & 0 & 0 & 1 \\
0 & 1 & -1 & 0 & -1 \\
0 & 0 & 1 & 1 & 0 \\
0 & 0 & 0 & 1 & -1
    }
    { 
      R_1 \to R_1 - 2*R_2
    }
    \eliminationstep
    {
1 & 0 & 2 & 0 & 3 \\
0 & 1 & -1 & 0 & -1 \\
0 & 0 & 1 & 1 & 0 \\
0 & 0 & 0 & 1 & -1
    }
    { 
      R_1 \to R_1 - 2*R_3\\
      R_2 \to R_2 + R_3
    }
    \eliminationstep
    {
1 & 0 & 0 & -2 & 3 \\
0 & 1 & 0 & 1 & -1 \\
0 & 0 & 1 & 1 & 0 \\
0 & 0 & 0 & 1 & -1
    }
    { 
      R_1 \to R_1 + 2*R_4\\
      R_2 \to R_2 - R_4\\
      R_3 \to R_3 - R_4
    }
    \eliminationstep
    {
1 & 0 & 0 & 0 & 1 \\
0 & 1 & 0 & 0 & 0 \\
0 & 0 & 1 & 0 & 1 \\
0 & 0 & 0 & 1 & -1
    }
    { 
      R_1 \to R_1 + 2*R_4\\
      R_2 \to R_2 - R_4\\
      R_3 \to R_3 - R_4
    }
\end{elimination}

מכוון שאין משתנה חופשי, למערכת הזאת יש פתרון יחיד מעל R  והוא  $\left(\begin{smallmatrix}1 \\ 0 \\ 1 \\ -1\end{smallmatrix}\right)$.

\subsection{ב}

נתחיל ישר משלב זה כי עד אז לא ביצענו שום פעולה אסורה מעל $Z_3$: 
\begin{elimination}[1]{4}{3em}{1.1}
  \eliminationstep
  {
    1 & 2 & 1 & 1 & 1 \\
    0 & 1 & -1 & 0 & -1 \\
    0 & 1 & 0 & 1 & -1 \\
    0 & 3 & 1 & 1 & 0
  }
  {
    \mod 3
  }
  \eliminationstep
  {
    1 & 2 & 1 & 1 & 1 \\
    0 & 1 & 2 & 0 & 2 \\
    0 & 1 & 0 & 1 & 2 \\
    0 & 0 & 1 & 1 & 0
  }
  {
    R_1 \to R_1 - 2*R_3
    R_3 \to R_3 - R_2
  }
  \eliminationstep
  {
    1 & 0 & 0 & 1 & 0 \\
    0 & 1 & 2 & 0 & 2 \\
    0 & 0 & 1 & 1 & 0 \\
    0 & 0 & 1 & 1 & 0
  }
  {
    R_2 \to R_2 - 2*R_3
    R_4 \to R_4 - R_3
  }
  \eliminationstep
  {
    1 & 0 & 0 & 1 & 0 \\
    0 & 1 & 0 & 1 & 2 \\
    0 & 0 & 1 & 1 & 0 \\
    0 & 0 & 0 & 0 & 0
  }
  {
  }
\end{elimination}
  \begin{align*}
    x + t &= 0 \implies x = -t \\
    y + t &= 2 \implies y = 2-t\\
    z + t &= 0 \implies z = -t\\
  \end{align*}
על $Z_3$ יש משתנה חופשי אחד והוא $t$ וכל שאר המשתנים קשוריים,
לכן יש למערכת הזאת יותר מפתרון, הפתרון הכללי הוא: 
\begin{equation*}
  \{-t, 2-t, -t, t\} 
\end{equation*}
מכוון ש-$t$ הוא מספר ב- $Z_3$, יש למערכת 3 פתרונות.

\section{שאלה 4}

\begin{elimination}[1]{3}{5em}{1.1}
  \eliminationstep
  {
1 & 2 & a & -3-a\\
1 & 2-a & -1 & 1-a\\
a & a & 1 & 6
  }
  {
    R_2 \to R_2 - R_1\\
    R_3 \to R_3 - a*R_1
  }
  \eliminationstep
  {
1 & 2 & a & -3-a\\
0 & -a & -1-a & 4\\
0 & -a & 1-a^2 & 6 + 3a + a^2
  }
  {
    R_3 \to R_3 - R_2
  }
  \eliminationstep
  {
1 & 2 & a & -3-a\\
0 & -a & -1-a & 4\\
0 & 0 & 2-a^2+a & 2 + 3a + a^2
  }
  {
    R_2 \to \frac{-R_2}{a} (a \neq 0)
  }
  \eliminationstep
  {
1 & 2 & a & -3-a\\
0 & 1 & \frac{1+a}{a} & \frac{-4}{a}\\
0 & 0 & 2-a^2+a & 2 + 3a + a^2
  }
  {
    R_3 \to \frac{R_3}{2-a^2+a} ((2-a^{2}+a) \neq 0 \implies a \neq -1, a \neq 2)
  }
  \eliminationstep
  {
1 & 2 & a & -3-a\\
0 & 1 & \frac{1+a}{a} & \frac{-4}{a}\\
0 & 0 & 1 & \frac{2 + 3a + a^2}{2-a^2+a}
  }
  {
  }
\end{elimination}
עבור  $a =2$: 

\begin{elimination}[1]{3}{3em}{1.1}
  \eliminationstep
  {
1 & 2 & 2 & 1 \\
0 & 1 & \frac{3}{2} & -2\\
0 & 0 & 0 & 12
  }
  {
  }
\end{elimination}
מקבלים שורת סתירה, לכן אין פתרון.
\\
עבור  $a=0$: 

\begin{elimination}[1]{3}{3em}{1.1}
  \eliminationstep
  {
1 & 2 & 0 & 3 \\
0 & 0 & -1 & 4 \\
0 & 0 & 2 & 2
  }
  {
  }
\end{elimination}
אין פתרון 
\\

עבור  $a=-1$

\begin{elimination}[1]{3}{3em}{1.1}
  \eliminationstep
  {
1 & 2 & -1 & -2 \\
0 & -1 & 0 & -4 \\
0 & 0 & 0 & 0 \\
  }
  {
  }
\end{elimination}
יש אין סוף פתרונות, הפתרון הכללי הוא:
\begin{equation*}
  \{s-10; 4; s\} 
\end{equation*}

\textbf{סיכום:}
\begin{itemize}
  \item יש פתרון יחיד עבור: $a\neq-1$, $a \neq 0$, $a \neq 2$ 
\item  יש אין סוף פתרונות עבור: $a=-1$ ופתרון הכללי הוא  $\{s-10; 4; s\}$
  \item אין פתרונות עבור: $a=0$, $a=2$
\end{itemize}

\section{שאלה 5}
   
\begin{elimination}[1]{4}{4em}{1.1}
  \eliminationstep
  {
    1 & a & a & a-b & b+1\\
    1 & a+1 & a+b & 2a-b & a+b+1 \\
    3 & 3a & 3a+b & 3a-b & 4b+3 \\
    1 & a & a & 0 & 2b
  }
  {
    R_2 \to R_2 - R_1\\
    R_3 \to R_3 - 3*R_1 \\
    R_4 \to R_4 - R_1
  }
  \eliminationstep
  {
    1 & a & a & a-b & b+1\\
    0 & 1 & b & a & a \\
    0 & 0 & b & 2b & b \\
    0 & 0 & 0 & b-a & b-1
  }
  {
    R_3 \to \frac{R_3}{b} (b \neq 0) \\
  }
  \eliminationstep
  {
    1 & a & a & a-b & b+1\\
    0 & 1 & b & a & a \\
    0 & 0 & 1 & 2 & 1 \\
    0 & 0 & 0 & b-a & b-1
  }
  {
    R_4 \to \frac{R_4}{b-a} (b \neq a)\\
  }
  \eliminationstep
  {
    1 & a & a & a-b & b+1\\
    0 & 1 & b & a & a \\
    0 & 0 & 1 & 2 & 1 \\
    0 & 0 & 0 & 1 & \frac{b-1}{b-a}
  }
  {
  }
\end{elimination}

עבור $b \neq 0$ ו-$b \neq a$ יש למערכת פתרון יחיד, נבדוק עכשיו את הערכים הבעיתיים.\\

עבור $b=0$
 \begin{elimination}[1]{4}{5em}{1.1}
  \eliminationstep
  {
    1 & a & a & a & 1\\
    0 & 1 & 0 & a & a \\
    0 & 0 & 0 & 0 & 0 \\
    0 & 0 & 0 & -a & -1
  }
  {
    R_4 \to -R_4\\
    R_3 \leftrightarrow R_4
  }
  \eliminationstep
  {
    1 & a & a & a & 1\\
    0 & 1 & 0 & a & a \\
    0 & 0 & 0 & a & 1 \\
    0 & 0 & 0 & 0 & 0 
  }
  {
    R_1 \to R_1 - R_3\\
    R_2 \to R_2 - R_3
  }
  \eliminationstep
  {
    1 & a & a & 0 & 0\\
    0 & 1 & 0 & 0 & a-1 \\
    0 & 0 & 0 & a & -1 \\
    0 & 0 & 0 & 0 & 0 
  }
  {
    R_3 \to \frac{R_3}{a} (a \neq 0) 
  }
  \eliminationstep
  {
    1 & a & a & 0 & 0\\
    0 & 1 & 0 & 0 & a-1 \\
    0 & 0 & 0 & 1 & \frac{1}{a} \\
    0 & 0 & 0 & 0 & 0 
  }
  {
    R_1 \to R_1 - a*R_2
  }
  \eliminationstep
  {
    1 & 0 & a & 0 & -a(a-1)\\
    0 & 1 & 0 & 0 & a-1 \\
    0 & 0 & 0 & 1 & \frac{1}{a} \\
    0 & 0 & 0 & 0 & 0 
  }
  {
  }
\end{elimination}
\begin{align*}
  x + az &= -a(a-1) \implies x = -a(a-1) - at\\
  y &= a - 1\\
  z &= t\\
  w &= \frac{1}{a}
\end{align*}
אם $a = 0$ אז $R_3$ שורת סתירה, ואין פתרון\\
אם $a \neq 0$, אז יש למערכת פתרון כללי, והוא:
\begin{equation*}
  \{-a(a-1) - at, a-1, t, \frac{1}{a}\}
\end{equation*}

עבור $a=b$

\begin{elimination}[1]{4}{3em}{1.1}
  \eliminationstep
  {
    1 & a & a & 0 & a+1\\
    0 & 1 & a & a & a \\
    0 & 0 & 1 & 2 & 1 \\
    0 & 0 & 0 & 0 & a-1
  }
  {
  }
\end{elimination}

אם $a \neq 1$ אז $R_4$ שורת סתירה ואין פתרון, אחרת קיים אין סוף פתרונות למערכת, פתרון הכללי הוא: 
\begin{align*}
  \{1-t, t, 1+2t, t\}
\end{align*}

\fbox{\parbox{\textwidth}{
סיכום: \\

אין פתרון:
\begin{itemize}
  \item $a=b \neq 1$
\end{itemize}

פתרון יחיד:
\begin{itemize}
  \item $b \neq 0$
    \item $a \neq b$
\end{itemize}

אין סוף פתרונות:
\begin{itemize}
  \item $b=0 \land a \neq 0$:$\{-a(a-1) - at, a-1, t, \frac{1}{a}\}$
    \item $a=b=1$: $\{1-t, t, 1+2t, t\}$
\end{itemize}
}}



\end{document}

