%
%  groups_theory.tex
%  preparation
%
%  Created by Eyal Shukrun on 08/22/20.
%  Copyright 2020. Eyal Shukrun. All rights reserved.
%

\documentclass[12pt]{article}

\usepackage{amsmath, amsfonts, amssymb}

\usepackage{cjhebrew}
\usepackage[utf8x]{inputenc}

\title{Groups Theory}
\date{\today}

\author{Eyal Shukrun}

\begin{document}
\maketitle

\section{Groupes Universels}
  
\subsection{Reels ($\mathbb{R}$)}
Tous les nombres réels.
\subsection{Rationels ($\mathbb{Q}$)}
Tous les nombres qui peuvent être exprimés en fraction.
\subsection{Integers ($\mathbb{Z}$)}
Tous les nombres entiers.
\subsection{Naturels ($\mathbb{N}$)}
Tous les nombres entiers positifs.
\\
\\
\textbf{Ainsi: $\mathbb{R} > \mathbb{Q} > \mathbb{Z} > \mathbb{N}$}
\\
\section{Notations}
\subsection{Est compris dans}
Si un élément a est compris dans un ensemble E, cela se note $a \in E$.
 
\subsection{Implications}
Si deux éléments ont un rapport logique (si A alors B), on utilise les flèches afin de le noter:\\
Si A alors B: $A \Rightarrow B$\\
Si B alors A: $A \Leftarrow B$\\
Double implication: $A \leftrightarrow B$\\
  
  
\section{Definitions}
  
\subsection{Groupe vide (\O)}
Il n'existe qu'un seul groupe vide, noté \O, il ne contient aucun élément.

\subsection{Ensembles egaux}
$A = B$ si tous les éléments de A sont presents dans B et inversement.  \\
\textbf{Attention:} Le nombre d'occurences n'importe pas.
 
\subsection{Ensemble compris}
On dit que A est compris dans B si tous les éléments de A se trouvent dans B.\\
{\em Notation}: $A \subseteq B$\\
\\
Ainsi, si $A \subseteq B$, alors $a \in B$ \\
\\
A est strictement compris dans B si $A \subseteq B$ et $A \neq B$\\
{\em Notation}: $A \subset B$\\

\subsection{Operations}
Une operation peut se produire uniquement entre deux ensembles, pas entre un ensemble et un élément.
\subsubsection{L'intersection}
L'intersection de deux ensembles A et B est l'ensemble de nombres se trouvant dans A et dans B.\\
{\em Notation:} $A \cap B$

\subsubsection{L'union}
L'union de deux ensembles A et B est l'ensemble de nombres se trouvant soit dans A soit dans B, soit dans les deux.\\
{\em Notation:} $A \cup B$

\subsection{Ensembles Étrangers}
Deux ensembles sont dits étrangers si $A \cap B = \O \\$ 

\subsection{Couples ordonnés}
Un couple ordonné est un groupe \textbf{ordonné} de deux nombres, il se note $(a,b)$.\\
\textbf{Attention:} $(a,b) \neq (b,a)$.\\
Ainsi: $(a,b) = (c,d)$ uniquement si $a = c$ et $b = d$ \\.

\subsection{Multiplication Cartesienne}
La multiplication cartesienne de deux ensembles $A$ et $B$ est l'ensemble de tous les couples ordonnés $(x,y)$ tel que $x \in A$ et $y \in B$.\\
Ainsi: $A*B = \{(x,y) | x \in A, y \in B \}$

\subsection{Power Set}
 Le power set d'un ensemble A est l'ensemble des sous ensembles de A.\\
 {\em Notation:} $P(A)$\\
 \textbf{Attention:} Les éléments de $P(A)$ sont eux mêmes des ensembles.\\
 \textbf{Attention:} P(A) contient toujours \O.\\
  

\section{Propriétés}
\subsection{Unions et Intersections}
Si $A \subseteq B$, alors:\\ 
$A \cap C \subseteq B \cap C$.\\
$A \cup C \subseteq B \cup C$.\\
\\
Si $A \subseteq B$ et $B \subseteq C$, alors $A \subseteq C$\\
\\
$A \cap (B \cup C) = (A \cap B) \cup (A \cap C)$\\
$A \cup (B \cap C) = (A \cup B) \cap (A \cup C)$\\
  


\section{Démonstrations}
\subsection{$A \subseteq B$}
Pour prouver que $A \subseteq B$, il faut prouver que n'import quel élément $a$ contenu dans $A$ est aussi contenu dans $B$.\\
Ainsi: Il faut prouver que $a \subseteq A \Rightarrow a \subseteq B$.\\

\subsection{$A = B$}
Pour prouver que $A = B$, il faut prouver que $A \subseteq B$ et que 
$B \subseteq A$.

\subsection{$ A \subseteq B \Rightarrow A \cap C \subseteq B \cap C$\\}
Soit $a \in A \cap C$, prouvons que $a \in B \cap C$.\\
\\
Si $a \in A \cap C$, alors $a \in A$ et $a \in C$.\\
Mais puisque $A \subseteq B$, alors $a \in B$.\\
Ainsi, puisque $a \in B$ et $a \in C$, $a \in B \cap C$.\\
\\
Donc $A \cap C \subseteq B \cap C$.

\subsection{$ A \subseteq B \Rightarrow A \cup C \subseteq B \cup C$\\}

Soit $a \in A \cup C$, prouvons que $a \in B \cup C$.\\
\\
Si $a \in A \cup C$, alors $a \in A$ ou $a \in C$.\\
Dans le cas ou $a \in C$, alors $a \in B \cup C$.\\
Dans le cas ou $a \in A$, puisque $A \subseteq B$ alors $a \in B$.\\
Ainsi $a \in B \subseteq C$.
\\
Donc $A \cup C \subseteq B \cup C$

\subsection{$R \subseteq S \Rightarrow R \cap S = R$ }
Pour démontrer que deux ensembles sont égaux, il faut demontrer que chaque élément du premier est présent dans le deuxième.\\
Ainsi, prouvons que:\\
1)$ R \cap S \subseteq R$\\
2)$ R \subseteq R \cap S$\\
\\
1) soit $a \in R \cap S$, alors $a \in R$ et $a \in S$, donc $a \in R$, et $R \cap S \subseteq R$.\\
2) soit $a \in R$, puisque $R \subseteq S$, alors $a \in S$, ainsi $a \in R$ et $a \in S$, donc $a \in R \cap S$, et $R \subseteq R \cap S$.
\\
Ainsi, $R \cap S = R$.

\subsection{$R \subseteq S \Rightarrow R \cup S = S$}
Prouvons que:\\
1)$R \cup S \subseteq S $\\
2)$S \subseteq R \cup S $\\
\\
1)Soit $a \in R \cup S$, alors $a \in S$ ou $a \in R$, si $a \in S$, il n'y a rien a démontrer, si $a \in R$, puisque $R \subseteq S$, alors $a \in S$, donc $R \cup S \subseteq S$. \\
2)Soit $a \in S$, alors par la definition de l'union, $a \in R \cup S$, donc $S \subseteq R \cup S$.
\\
Ainsi: $R \cup S = S$.
\\
\\
\\
\textbf{FIN.}
\end{document}

