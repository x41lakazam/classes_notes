%
%  class_notes_2610.tex
%  latex_files
%
%  Created by Eyal Shukrun on 10/26/20.
%  Copyright 2020. Eyal Shukrun. All rights reserved.
%


\documentclass{article}
\usepackage[pdftex]{graphicx}
\usepackage[utf8x]{inputenc}
\usepackage{pslatex}
\usepackage{amsmath, amsfonts, amssymb}
\usepackage[hebrew,english]{babel}
\usepackage{cjhebrew}
\usepackage{graphicx}
\usepackage{mymacros}


\title{Titre}
\author{Eyal Shukrun}

\begin{document}
\maketitle

\section{Equation avec un parametre ($x^{2} + ay + z = 0$)}
  
\paragraph{Trouver les valeurs de a pour lesquelles le systeme a X solutions}
I.
\begin{itemize}
  \item Ecrire la matrice avec le paramètre et la mettre en escalier
  \item Trouver toutes les valeurs pour lesquelles il n'y a pas qu'une seule solution
  \item Les essayer pour voir ce qu'il se passe
\end{itemize}

Si la matrice est en escalier, alors normalement il n'y a qu'une seule solution. Mais si les ouvreurs contiennent le paramètre, alors il peut y avoir une valeur du paramètre pour laquelle l'ouvreur devient 0.. alors.. attention..

\section{Systeme d'equation quelquonque}
\paragraph{Trouver la solution generale}
\begin{itemize}
  \item Mettre la matrice en canonique
\end{itemize}
 
\section{Systeme d'equation sur un champs specifique}
On peut faire passer une matriuce d'un champs a l'autre tant qu'on ne fait multiplie pas par 0. Par exemple sur un ensemble $Z_n$ il ne faut pas multiplier par un multiple de $n$. 
\section{Matrice non proportionelle}
Par exemple, le $2-m$ et le $m^{2}$ sont problematiques ici:
\begin{equation}
  \begin{pmatrix} 
    1 & 2 & 3 \\
    0 & 2-m & m-2 \\
    0 & m^2 & 3m-2 
  \end{pmatrix} 
\end{equation}

Multiplier la ligne problematique par l'inverse de l'ouvreur (donc il devient 1), \textbf{attention, a partir de la, on supprime une solution car le dénominateur ne peut pas être 0, il faut donc tester cette solution a part)}.

\section{Est ce que un vecteur est la combinaison linéaire de plusieurs vecteurs?}
Regarder si le système obtenu en créant les equations est solvable.

\end{document}

