%
%  1_linear_equations_systems.tex
%  latex_files
%
%  Created by Eyal Shukrun on 09/08/20.
%  Copyright 2020. Eyal Shukrun. All rights reserved.
%

\RequirePackage[l2tabu, orthodox]{nag}
\documentclass[12pt]{article}

\usepackage{amssymb,amsmath,verbatim,graphicx,microtype,upquote,units,booktabs,siunitx,xcolor}
\usepackage{cjhebrew}
\usepackage{siunitx}
\usepackage{array}
\usepackage{mymacros}


\title{Linear Equations Systems}
\date{\today}
\author{Eyal Shukrun}

\begin{document}
\section{Ensemble}
(F, +, *) est un ensemble si il respecte ces 6 axiomes: \\

\begin{itemize}
  \item F est fermé sur + et *. 
  \item Association de + et *
  \item + et * sont inversables
  \item Il existe des elements neutres pour + et * \textbf{uniques et non égaux}.
  \item * se distribue sur +
  \item $ \forall x \neq 0$ il existe $x^{-1}$
\end{itemize}

\subsection{$Z_n$}
L'ensemble $Z_n$ est un ensemble qui contient tous les elements de $Z$ de 0 a $n$.
\begin{enumerate}
  \item $a +_F b = a+b \mod n$ 
  \item $a *_F b = a*b \mod n$ 
\end{enumerate}

\subsection{Comment trouver rapidement le résultat}
Faire le calcul normal, puis ajouter/enlever $n$ jusqu'a tomber sur un chiffre dans l'ensemble $\{0, \ldots, n\}$.
  
  

\subsection{Propriétés des scalaires}

Tous les nombres d'un champs sont nommés scalaires.  

\begin{itemize}
 \item 0 et 1 sont uniques
 \item Pour chaque nombre il existe un seul nombre inverse et contraire

 \item $a*b = 0 \implies a=0 \lor b=0$
\end{itemize} 

\subsection{Demonstrations}
 \paragraph{Prouvons que $a*b = 0 \implies a=0 \lor b=0$:}
   

Si $a \neq 0 \land b \neq 0$ alors $a*b=0 \equiv a^{-1}*a*b = a^{-1}*0 \equiv b = 0 \equiv F$, donc a ou b doit etre égal a 0.  

\subsection{Chemins de solutions}

\paragraph{Comment trouver le 0 et le 1:}
On a juste a resoudre b dans les equations $a+b = a$ et $a*b = a$, il doit n'y avoir qu'une seule solution de b pour tous les a.  

\paragraph{Comment trouver l'inverse et le contraire:} 
Il faut d'abord connaitre le 0 et le 1, puis resoudre b dans les equations $a+b = 0$ et $a*b = 1$, il ne doit ici aussi n'y avoir qu'une seule solution, (attention, b sera egal a $-a$, pas $a$, et les symboles +, *, 0 et 1 sont dependants du champs duquel on parle).
  
\section{Equations linéaires}
Une équation Linéaire est une équation polynomiale qui ne contient pas de multiplications d'éléments ($xy$) ou d'éléments seuls avec une puissance ($z^3$). Ainsi, elle est de ce format:
\[
  a_1 * x_1  + a_2 * x_2 + \ldots + a_n * x_n = b
.\]

\subsection{Appellations}
\begin{itemize}
  \item $x$ - variables
  \item $a$ Mekadmim
  \item Mekadmim hofshiim$b$ 
  \item  Equation zero - Equation ou tous les mekadmim sont $0_F$
  \item Solution privee - solutions sous forme de nombres
  \item Solution generale - solutions pour chaque variable sous forme de fonction avec des parametres, par exemple $(\frac{7}{2}+\frac{3}{2}t, t)$, on peut egalement la factoriser pour obtenir $(\frac{7}{2}, 0) + t(\frac{3}{2}, 1)$

\end{itemize}

\subsection{Propriétés}
\begin{enumerate}
  \item Si une fonction linéaire a des paramètres provenant d'un ensemble A et une solution sur un autre ensemble B, alors elle a aussi une solution sur A.
\end{enumerate}

\section{Systeme d'equations lineaires}

\subsection{Appellations}
\begin{itemize}
  \item $(v_1, \ldots, v_n) \epsilon  F^n$ est la solution de l'equation si $(x_1, \ldots, x_n) = (v_1, \ldots, v_n)$ resoud toutes les equations.
  \item  Ordre du systeme $(m, n)$ - $m$ est le nombre d'equations, et $n$ est le nombre de variables.  
    \item Forme standard: Comme une matrice
  \item \cjRL{`qbyt}: Systeme solvable.  
    \item  Equation zero: $0*x_1 + 0*x_2 + 0*x_3 = 0$, n'importe quel groupe la resoud.
    \item Systeme homogene: tous les mitkadmim hofshiim sont $0_F$ (il est forcement solvable).
\end{itemize}

\subsection{Propriétés}
\begin{enumerate}
  \item Si une equation lineaire ressemble a $0*x_1 + 0*x_2 + 0*x_3 = b(\neq_0)$ alors elle n'a pas de solution.
  \item  Un systeme homogene est forcement solvable par la solution triviale $(0, 0,\ldots, 0)$.
  \item Dans un systeme homogene, si $c$ est une solution, alors  $sc$ (s scalaire) l'est aussi, et si  $d$ est une solution, alors $c + d$ aussi
\end{enumerate}

\subsection{Systemes equivalents}

On peut appliquer les operations suivantes sur un systeme sans le modifier:
\begin{enumerate}
  \item Changer les equations de place
  \item Additioner une equation avec une autre (ou avec une multiplication d'une autre par un scalaire)
  \item Multiplier une equation entiere par un scalaire qui n'est pas $0_F$
\end{enumerate}

\textbf{Attention: Ne pas faire d'opération $R_A$ - $R_B$ si $R_A$ a des 0 sur lesquels $R_B$ a des chiffres !}

\section{Matrices de coefficients}
Avec cette matrice, on peut resoudre l'equation, en essayant d'avoir des parametres que sur une sur ligne.  

\subsection{Exemple}

\begin{align*}
  a_{1_1}x_1 + a_{1_2}x_2 + \ldots + a_{1_n}x_n &= b_1\\
  a_{2_1}x_1 + a_{2_2}x_2 + \ldots + a_{2_n}x_n &= b_2\\
  .         .           . \\
  a_{3_n}x_1 + a_{3_n}x_2 + \ldots + a_{n_n}x_n &= b_n\\
\end{align*}

Donne: 

\begin{elimination}[1]{4}{1.75em}{1.1}
    \eliminationstep
    {
      a_{1_1} & a_{1_2}  &  \ldots & a_{1_n} & b_1\\
    a_{n_1} &a_{2_2}  &  \ldots & a_{2_n} & b_2\\
      \ldots & \ldots &  \ldots & \ldots & \ldots \\
      a_{m_1} &a_{m_2}  &  \ldots & a_{m_n} & b_m\\
    }
    {
    }
\end{elimination}

\subsection{Appellations}
\begin{itemize}
  \item Element ouvreur (\cjRL{'ybr hptw.h}): Premier element de chaque ligne.
  \item  Variables liées et variables libres: une variable liée est une variable qui est attachée a un element ouvreur au moins une fois dans la matrice (on parle ici du $x$). Une fois que la matrice est en escalier, on peut exprimer chaque variable liée a l'aide des variables libres.
  \item Matrice escalier (\cjRL{m.try.sh mdrgwt}): Toutes les lignes zero sont en bas, et chaque element ouvreur et a droite de celui de la ligne d'au dessus. Le systeme est donc appelé système escalier (\cjRL{m`rkt mdwrg}).
    \item  Matrice canonique: Tous les elements ouvreurs sont des 1 et ils sont seuls dans leur colonne. Il n'y a qu'une et unique matrice canonique pour un système linéaire. On peut donc s'en servir pour regarder si deux matrices sont équivalentes. 
    \item  Matrice identité (\cjRL{y.hydh}) d'ordre n: Matrice de forme $n*n$ avec que des 0 sauf des 1 dans la diagonale principale (\cjRL{'lkswN hr'/sy}).
    \item Système homogène: Tous les mekadmim hofshiim ($b$) sont 0, ces systèmes ont au moins une solution (la solution \textbf{triviale} $(0, \ldots, 0)$), si le nombre de variables est plus grand que le nombre d'équations alors le systeme a aussi une solution non triviale. Un système homogène peut être représenté par une matrice réduite (une matrice qui ne contient pas les mekadmim hofshiim ($b$)).
\end{itemize}

\subsection{Mettre une matrice en escalier}
\begin{enumerate}
  \item Mettre tout en haut la ligne avec l'ouvreur le plus a gauche
    \item Utiliser cette ligne du haut pour tondre cette colonne, de sorte a ce que la ligne du haut soit la seule a avoir un ouvreur sur cette colonne
      \item Recommencer en ignorant la premiere ligne
\end{enumerate}

\subsection{Mettre une matrice en canonique}
\begin{enumerate}
  \item La mettre en escalier
  \item Transformer l'ouvreur de la dernière ligne en 1 en multipliant la ligne par son inverse
  \item Tondre toute la colonne grace a lui
\end{enumerate}
  

\subsection{Chemins de solution}
\begin{itemize}
  \item Regarder si deux matrices sont équivalentes: les réduire a leur forme canonique et regarder si elles sont pareilles. 
\end{itemize}

\section{Nombre de solutions d'un système linéaire}

Un système linéaire peut soit ne pas avoir de solutions, soit n'en avoir qu'une seule et unique, soit en avoir une infinité (si l'ensemble de definition est fini). 
\begin{enumerate}
  \item Pas de solution: On peut le determiner si une matrice a une ligne de contradiction, si une matrice escalier n'a pas de ligne contradictoire, normalement elle a au moins une solution (\cjRL{`qbyt})
  \item Une solution: Toutes les variables sont des variables liées donc la matrice canonique reduite (\cjRL{m.swm.smt}) est une matrice identité. D'ailleurs, \textbf{si une matrice canonique réduite est une matrice identité, alors tous les systèmes dont elle peut être la matrice (peu importe les b) ont une solution unique.} On en déduit que pour un système homogène la matrice réduite a soit des lignes de 0 (une solution ou plus) ou alors est une matrice identité (donc solution unique triviale$^{[1.14.4]}$).
    \item Si elle a des variables libres, alors elle a un nombre infini de solutions sur un ensemble infini, et $| F |^{libres}$ solutions sur un ensemble  $F$ fini.
\end{enumerate}

  
  

\end{document}

