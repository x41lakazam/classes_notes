%
%  2_nuples.tex
%  latex_files
%
%  Created by Eyal Shukrun on 10/22/20.
%  Copyright 2020. Eyal Shukrun. All rights reserved.
%

\RequirePackage[l2tabu, orthodox]{nag}
\documentclass[12pt]{article}

\usepackage{amssymb,amsmath,verbatim,graphicx,microtype,upquote,units,booktabs,siunitx,xcolor}
\usepackage{cjhebrew}
\usepackage{siunitx}
\usepackage{array}
\usepackage{mymacros}



\title{Espaces vectoriels}
\author{Eyal Shukrun}

\begin{document}
\maketitle

\section{Rappel}
Soit deux vecteurs a et b et un scalaire f.
  \begin{itemize}
    \item $a+b = (a_1+b+1, a_2+b_2, \ldots, a_n + b_n)$
    \item $a*f = (a_1*f, a_2*f, \ldots, a_n*f)$
  \end{itemize}  

\section{Espace vectoriel}
L'ensemble des n-uples (\cjRL{ywt}\textbf{-n}, ensemble des groupes de taille $n$) d'un champ $F$ est noté $F^n$. $F^n$ est un espace vectoriel, c'est a dire un ensemble de vecteurs que l'on peut additioner entre eux et multiplier par un scalaire.  
  
\subsection{Proprietés}
\begin{enumerate}
  \item Si \textbf{a} et \textbf{b} sont des nuples de $F$ et $f$ un scalaire de $F$ alors $a+b \in F$ et $a*f \in F$
\end{enumerate}
 
\subsection{Vecteurs}
Défini par sa longueur et sa direction. 

\subsubsection{Proprietés}
\begin{enumerate}
  \item Additionner deux vecteur $(a_1, a_2)$ et $(b_1, b_2)$: $(a_1+b_1, a_2+b_2)$. Le resultat est le vecteur qui relie la queue de a a la tete de b. Le resultat de $a-b$ est le vecteur qui relie leurs deux tetes.     
  \item Multiplier un vecteur $(a_1, a_2)$ par un scalaire $t$: $(a_1*t, a_2*t)$. Le résultat allonge ou raccourcit le vecteur, si $t < 0$ alors le vecteur change de direction.
\end{enumerate}

 
\subsection{Representation graphique de combinaisons linéaires}

Nous ne parlerons ici que de systèmes linéaires a 2 ou 3 inconnues (car ils se représentent en 2D ou 3D). 

\subsubsection{Intuition $^{[p.169]}$}
  
Si on prend un vecteur $a$ dans un plan en 3D, disons $(1,1,1)$, nous avons donc une flèche. Si on le multiplie par 2, la flèche devient 2 fois plus longue, ainsi si on trace tous les points qui sont touchés par $t*a$ en faisant varier  $t$, on obtient une droite.

Maintenant, prenons un vecteur $x$ qui est l'addition de deux vecteurs $a+b$ (pas equivalents), disons $a=(1,0,0)$ et $b=(0,1,0)$, et faisons varier ces deux vecteurs, ainsi $x= t*a + s*b$, on sait deja que $t*a$ est une droite (et que $s*b$ aussi), pour chaque point sur la droite, on peut donc faire varier $s$ pour toucher n'importe quel point sur la droite $s*b$. Si on dessine tous les points pour chaque valeurs de $t$ et chaque valeurs de $s$, on obtient un plan.

Si on ajoute un troisième vecteur a cette somme, et que $x = t*a + s*b + r*c$, en faisant varier t, s et r nous pourrons être capables de toucher n'importe quel point dans l'espace.  

Attention, pour pouvoir jouer avec un vecteur, il faut qu'il ait un coefficient variable. Par exemple, le vecteur $x = t*a + s*b + c$ n'est qu'un plan, avec un shift de $c$, car  $c$ ne peut pas varier, tout comme $x = t*a + b$ n'est qu'une droite.

On peut donc connaitre la representation graphique des solutions d'une equation grace a sa matrice escalier et a son nombre de lignes non-zeros. Plus il y a de lignes non-zeros, moins il y a de solutions (car plus de contraintes). Visuellement, le nombre de lignes zeros correspond au nombre de vecteurs que l'on peut faire varier. Ainsi pour un système a 3 inconnues:
\begin{enumerate}
  \item Si elle n'a qu'une ligne non-zeros, la solution est un plan.
  \item  Si elle a deux lignes non-zeros, la solution est une droite.
  \item Si elle a trois lignes non-zeros, la solution est un point.
    \item Si toutes ses lignes sont des zeros, alors n'importe quel vecteur resoud l'équation, donc la solution est tout l'espace.
\end{enumerate}

Évidemment, tout plan ou droite en plus de 3 dimensions a les mêmes propriétés. Plus généralement, une droite est exprimée sous la forme $t*a$ (a est un vecteur) et un plan sous la forme $t*a + s*b$ (a et b sont des vecteurs), ce plan est appelé le {\em span} (\cjRL{hpr/s}) de a et b.$^{[p.173]}$ 

\subsection{Combinaisons linéaires}
Une combinaison linéaire est la somme de plusieurs vecteurs, eux mêmes multipliés par des coefficients, ainsi soit $s_n$ des scalaires et $a_n$ des vecteurs, la combinaison linéaire s'exprime comme ceci:

\begin{equation}
  \sum_{i=1}^{n} a_i * s_i
\end{equation}  

\subsubsection{Combinaison de solutions d'un système}

Si $(a_1, a_2, \ldots, a_n)$ sont n solutions d'un système \textbf{homogène}, alors pour chaque solution, on peut la multiplier par un scalaire, cela donnera toujours 0. Donc pour tout vecteur $(s_1, s_2, \ldots, s_n) $, $(s_1a_1 + s_2a_2 + \ldots + s_na_n)$ est aussi une solution.  

Si d'un autre coté on a un système linéaire, le vecteur $b$ est une combinaison linéaire des vecteurs $(a_1, a_2, \ldots, a_n)$ et des coefficients $(s_1, s_2, \ldots, s_n)$ (les x, y, z etc..).$^{[p.180]}$

\subsubsection{Vecteur indépendants ( \cjRL{tlwt lyn'ryt})}
Deux vecteurs a et b (ou plus) sont indépendants si $\alpha*a + \beta*b = \textbf{0}$ s'obtient uniquement avec des coefficients de combinaisons égaux a 0. Si au moins un scalaire n'est pas 0 alors les vecteurs ne sont pas indépendants ( \cjRL{tlwy lyn'ryt} ). \\

Pour verifier si des vecteurs sont indépendants, les mettre dans une matrice a la verticale (la colonne des solutions est une colonne de 0). Si il y a plus d'une solution (donc si il y a des mishtanim hofshiim), ils ne sont pas indépendants (sinon, cela veut dire que la seule solution est la solution triviale et qu'ils sont indépendants).\\

Géométriquement parlant, des vecteurs indépendants ne sont pas dans la même direction (pas superposes), car si deux vecteurs sont superposes, on peut revenir a 0 en soustrayant l'un a l'autre, si ils ne sont pas dans la même direction, on ne peut pas les faire revenir a 0. \\

Dans un groupe de vecteurs, il suffit qu'un seul ne soit pas indépendant des autres pour que tout le groupe soit dependant. Il en decoule que:
\begin{itemize}
  \item Un groupe de vecteurs qui contient un sous groupe dependant est lui aussi dependant, et un sous groupe d'un groupe indépendant est lui aussi indépendant.$^{[p.186]}$ 
  \item Des vecteurs $ a_1, \ldots, a_k$ d'un espace $F^{n}$ sont forcement dependants si $k > n$, et si ils sont independants alors forcement  $k < n$ (car si il y a plus de vecteurs que de dimensions, et vu que l'on peut toucher n'importe quel point avec les n dimensions, on pourra forcement toucher le dernier vecteur).$^{[p.189]}$ 

\end{itemize}

\section{Base vectorielle}
\subsection{Base standard}
La base standard (\cjRL{bsys hs.tndr.ty}) d'un espace vectoriel composée de $ e_1, e_2, \ldots, e_n$ est l'ensemble des vecteurs de base d'un espace (vecteur avec un seul 1 et que des 0). La position du 1 dans le vecteur est l'index du $e$, par exemple, dans  $F^{5}$, $ e_3 = (0,0,1,0,0)$. Le span d'une base correspond a l'espace vectoriel entier (\cjRL{hbsys hs.tndr.ty pwr/s 't $F^{n}$}).\\
N'importe quel point de l'espace est une combinaison de la base standard. \\

\subsection{Base}
Si une suite (ou un ensemble) de vecteurs $ a_1, \ldots, a_k$ span un espace $F^{n}$, alors $k \ge n$, et si en plus cette série est indépendante alors $k = n$.$^{[p.192]}$\\
Donc si une suite de n vecteurs (ou un ensemble de n vecteurs différents) span un espace ou est indépendante, alors c'est la base de $F^{n}$.
Une série de vecteurs qui span un espace est appelée la base ordonnée (\cjRL{bsys sdwr})$^{[p.193]}$\\
L'ensemble des vecteurs de la serie est appelé  base (\cjRL{bsys}) si il est indépendant et qu'il span l'espace.\\
Un ensemble d exactement $n$ vecteurs indépendants span forcement $F^{n}$ (pas vrai si $k \neq n$).$^{[p.194]}$ 

Dans chaque base de $F^{n}$ il y a exactement $n$ vecteurs différents.\\

Chaque vecteur de $F^{n}$ correspond a une seule et unique combinaison de la base (la suite, pas l'ensemble, car l'ensemble n'est pas ordonné). \\

\subsubsection{Difference entre une suite et un ensemble}
Dans un ensemble, il ne peut pas y avoir de doublon, donc pour tout vecteur $v \neq 0$, l'ensemble {v,v} est indépendant, mais la suite $v,v$ ne l'est pas. Il n'y a donc pas de difference sauf si deux elements sont égaux.
  
\subsubsection{Prouver qu'un ensemble de vecteurs est une base}
Pour prouver qu'un ensemble (ou une suite) de vecteurs est une base il faut montrer que n'importe quel vecteur $b$ est une combinaison linéaire de ces vecteurs (voir p.195), ou montrer que le système homogène n'a pas de solution non triviale (en verifiant si sa matrice réduite est équivalente a la matrice identité, voir p.196). 

\subsection{Parallèle avec des systèmes}
Les vecteurs de la base d'un espace forment un système homogène a solution triviale: 
\begin{equation}
  s_1*a_1 + \ldots + s_n * a_n = 0
\end{equation}
Donc il n'existe qu'une seule et unique solution au système $s_1*a_1 + \ldots + s_n * a_n = b$$^{[p.198]}$ 
  

\end{document}
