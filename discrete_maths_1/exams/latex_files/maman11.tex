%
%  maman11.tex
%  latex_files
%
%  Created by Eyal Shukrun on 11/01/20.
%  Copyright 2020. Eyal Shukrun. All rights reserved.
%

\documentclass{article}
\usepackage[pdftex]{graphicx}
\usepackage{pslatex}
\usepackage{amsmath, amsfonts, amssymb}
\usepackage{graphicx}
\usepackage{mymacros}

\usepackage[utf8x]{inputenc}
\usepackage[hebrew,english]{babel}
\usepackage[top=2cm,bottom=2cm,left=2.5cm,right=2cm]{geometry}
\selectlanguage{english}

\title{}
\author{Eyal Shukrun}

\begin{document}
\maketitle

\selectlanguage{hebrew}
\section{שאלה 1}
\begin{itemize}
  \item א - נכון
  \item ב - לא נכון
  \item ג - לא נכון
  \item ד - לא נכון
  \item ה - נכון
  \item ו - לא נכון
  \item ג - נכון
  \item ח - נכון
\end{itemize}


\section{שאלה 2}
\section{שאלה 3}
\subsection{א}

\subsection{ב}
מתוך ההנחה ש-$A^{c} \Delta B = A \Delta  C $ נוכיח שנובע $C = B^{c} $.
למען ההוכחה נשתמש בשני הוכחות:
\begin{itemize}
  \item שאלה 83: $A \Delta U = A^{c} $
  \item שאלה 23: $A \Delta B = A \Delta C \implies B \Delta C$ והפעולה של הפרש סימטרי קיבוצית.
\end{itemize}

לפי שאלה 83: $A^{c} \Delta  B = (A \Delta  U) \Delta B$\\
לפי שאלה 23: $(A \Delta U) \Delta B = A \Delta (U \Delta B)$\\
שוב לפי 83: $A \Delta (U \Delta B) = A \Delta B^{c}$\\
הגענו לשוויון: $A \Delta B^{c} = A \Delta C$, לכן לפי שאלה 23 נובע כי $A \Delta B^{c} = A \Delta C \implies  B^{c}  = C$, שזה מה שהיה צריך להוכיח. 

\subsection{ג}

מתוך הנחה ש-$x \in (A \cap  B) \setminus C$, נובע כי $x \in (A \cap B) $ וכי $x \not\in C $, אך לפי הגדרת ההפרש הסימטרי: $A \Delta B = (A \cup  B) \setminus  (A \cap  B)$, לכן בהכרח $x \notin A \Delta B$, כי $x \in (A \cap B)$.  \\
מהנתון, $x \not\in C $, לכן אין צורך לבדוק ש-$x \not\in A \Delta C$ וש-$x \not\in B \Delta C$.\\
הוכחנו ש-$x \in (A \cap  B) \setminus C \implies x \not\in A \Delta B \Delta C$.

\section{שאלה 4}
\subsection{א}
משמעות הקבוצה  $A_n^{c}$ היא: כל המספרים הטבעיים הגדולים מ-$n$. לכן $\forall n \in N (A_0^{c} \subseteq A_n^{c})$.\\
לפי הגדרת האיחוד: $\underset{n=0}{\overset{\infty}{\cup }} A_n^{c} = A_0^{c} = N \setminus \{0\}$
  
\subsection{ב}

משמעות הקבוצה פה לא השתנה, אך הפעולה היא חיתוך ולא איחוד.\\
לכן רק איבר שנמצא בכל הקבוצות יהיה בחיתוך. אך לכל $x \in N$ קיים קבוצה  $A_x^{c}$ שמשמעותה $\{a \in N | a > x \}$ ולכן $x \not\in A_x^{c}$.\\
מזה נובע ש-$ \underset{n=0}{\overset{\infty}{\cap }} A_n = \varnothing$


\subsection{ג}
משמעות $(A_{2n} \setminus A_n)$ היא: כל האיברים מ-$n$ ל-$2n$, בלשון אחר: $\{x \in  N^{*} | n < x \le 2n \}$, 
אפס הוא מיוחד מכוון ש-$0 = 2*0$, לכן אפשר להוציא אותו ממשמעות הכללי של הקבוצה,
כי כמובן ש-$A_0 \setminus  A_0$ היא קבוצה ריקה. לכל $x$ פרט ל-1 אפשר למצוא $n \in N$ 
כך ש-$n < x \le 2n$.\\
ולכן $\underset{n=0}{\overset{\infty}{\cup }} (A_{2n} \setminus  A_n)$ מכילה את כל
המספרים הטבעיים הגדולים מ-1.\\
בלשון אחר: $\underset{n=0}{\overset{\infty}{\cup }} (A_{2n} \setminus  A_n) = N \setminus \{0,1\}$\\
לכן היא לא שווה לאחת מהקבוצות $N$ ,  $N \setminus \{0\}$ , $ \varnothing $.


\subsection{ד}
כפי שכבר ראינו, $A_n^{c}$ היא קבוצה של כל המספרים הטבעיים הגדולים מ-$n$, כלומר ${x \in  N | x > n}$\\
ולפי הגדרת $A$: משמעות $A_{n+1}$ היא כל המספרים הטבעיים מ-0 עד $n+1$, כלומר $\{x \in N | x \le n+1\}$.\\
לכן לכל $n \in N$, $(A_{n+1} \cap A_n^{c}) = \{n+1\}$, ולכן לכל $x \in N^{*}$ 
קיים: $x \in  (A_{x} \cap A_{x-1}^{c})$.\\
לכן האיחוד הזה היא קבוצה של כל המספריים הטבעיים הגדולים מ-0, כלומר:\\
$ \underset{n=0}{\overset{\infty}{\cup }} A_n = N \setminus \{0\}$



\end{document}

