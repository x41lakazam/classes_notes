%
%  2_rapports.tex
%  latex_files
%
%  Created by Eyal Shukrun on 10/25/20.
%  Copyright 2020. Eyal Shukrun. All rights reserved.
%


\documentclass{article}
\usepackage[pdftex]{graphicx}
\usepackage[utf8x]{inputenc}
\usepackage{pslatex}
\usepackage{amsmath, amsfonts, amssymb}
\usepackage[hebrew,english]{babel}
\usepackage{cjhebrew}
\usepackage{graphicx}
\usepackage{mymacros}
\usepackage{tikz-cd}


\title{Relations - \cjRL{y.hsym} }
\author{Eyal Shukrun}

\begin{document}
\maketitle

\section{Tuples - \cjRL{zwg sdwr}}
Liste de $n$ nombres (ordonnée).

\subsection{Multiplication cartesienne}
Multiplication cartesienne de $n$ ensembles: Tous les n-uples possibles composés des elements de chaque ensemble.
\begin{align*}
  A * B &= \{<a,b> | a \in A \land b \in B\}\\
  A * B * C &= \{<a,b,c> | a \in A \land b \in B \land c \in C\}\\
.\end{align*}

Plus generalement: 
\begin{equation}
  A_1 * \ldots * A_n = \underset{i=1}{\overset{n}{\times}} A_i 
\end{equation}

Donc:
\begin{equation}
  (A*B)*C = \{<<a,b>, c>\}
\end{equation}

Ainsi la multiplication caresienne n'est pas associative.  

\subsection{Propriétés}
\begin{itemize}
  \item $A * (B \cup C) = (A*B) \cup (A*C)$
  \item $A * (B \cap C) = (A*B) \cap (A*C)$
  \item $A * (B \setminus  C) = (A*B) \setminus (A*C)$
\end{itemize}
\section{Relations}
 
Une relation $R$ est un ensemble compose de n-uples (couples si c'est une relation a deux places (\cjRL{dw mqwmy}), 3 si c'est une relation a trois places etc). On l'appelle relation de A a B (ou relation sur A si elle est de A a A).\\
Cette relation est l'ensemble des n-uples qui respectent une certaine règle (définie par nous), donc $R \in P(A*B)$
. Pour exprimer un n-uple $<a,b>$ qui respecte R on peut utiliser $<a,b> \in R$ ou $aRb$. \\
Pour representer une relation on peut utiliser soit une matrice ou chaque colonne est un n-uple qui respecte la relation (l'élément en haut est l'élément de gauche du n-uple), soit par un tableau (les titres des lignes sont les elements de gauche, voir p.77), soit un diagramme a flèches, par exemple pour la relation $R = \{<1,a>, <2, a>\}$:


\begin{tikzcd}
  1 \arrow[r] & a \\
  2 \arrow[ru] & b \\
               & c \\
\end{tikzcd}


\end{document}

